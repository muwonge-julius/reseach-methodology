\documentclass{article}
\begin{document}
\author{Muwonge Julius REG 15/U/8595/PS}
\title{HOW TO BUY A KIKOMANDO IN KIKONI WITHOUT BEING SEEN.}
\maketitle
\section{Executive summary}{It has been discovered that it is a self learnt survival mechanism for every campuser who lives at kikoni to learn eating kikomando. However over time, experience has shown that eating and buying kikomando is a shameful habit to the victims. As a result, several tactics of avoiding exposure have been invented by students and are as explained below.
}
\section{Introduction}{Earlier before kikomando was nowhere to be seen that is to say in 1930s but later as the number of students increased, a number of survival mechanisms were put in place through which kikomando was unveiled. Kikomando is a mixture of cut chapatti and beans and it is much more attended by the university students. Kikomando came after students failing to live on expensive bought food sold at kikoni so as a result kikomando which is cheaper was invented.
}
\section{Background}{As a result of the gradual increase in the numbers of students that find life in feeding on kikomando, a number of buying and eating tactics have been put in place and without delay, here I present to you the technics.
}

{The first technic is buying from isolated selling points. This technic has worked for several students in different magnitudes. Here students don’t buy from selling points that are along kikoni streets but instead go for selling points that are a bit a distance from other student’s exposure for example mulokole’s selling point behind sunways hostel.
}

{Buying at night. This has worked for so many that can either surrender lunch, by surrender lunch I simply mean not eating lunch or eat something else other than kikomando for example eating food or buying a snack and feast on it. This has resulted to night collision at the kikomando selling points as a result of big crowds around the kikomando selling points.
}

{Hiding identity. In these technic students fight by all means to hide their really identity that is to say through putting on big clothes like jumpers or putting on a top(cape) in order to be not easily identified.
}

{The one eye technic.
It is a technic most especially used by students who have hopes of falling in love one time. This is seen being done by especially first years and some second years. What happens in this method is, students tend to approach the kikomando selling points but with one eye closed. Briefly what i mean by one eye being closed is that, students approach the kikomando selling points with strong consolation feelings like “no one sees me, no one cares” but without displaying their faces openly to the public, so what they do is, they keep looking down on the ground as there are making orders. And they don’t normally move their heads for face exposure sideways. So with that, they tend to not even care about who their neighbors are at the selling point. And slowly by slowly they cling on as their orders are being implemented. Time after receiving their kikomando, they move slowly to their rooms where they eat it from.
}

{Buying through sending someone. Here one person can buy for over four people. This is seen being done by friends that have either lunch or supper together especially hostel friends and course mates.
}

{Making an order and service delivered. Here a student makes an order and the kikomando is delivered to his place of comfort by the seller. 
}
\section{Findings and conclusion}{Kikomando is a daily food attended by university students.
}

{Some sellers in order to have a big market, they have gone on ahead to put sacraments on kikomando for example salads, cabbage, and other vegetables.}

{Every other year a number of kikomando selling places are put in place.}

{Every other year there are new kikomando customers.}

{Kikomando is not a good food to rely on every day as it may cause constipation.}

\section{lessons}{Kikomando is cheaper and a survival mechanism.}
\section{conclusion}{Moderate eating of Kikomando is a survival mechanism for a university student.
}

\end{document}